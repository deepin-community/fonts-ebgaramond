\documentclass{article}

\usepackage{amsmath}
\usepackage{amssymb}

\usepackage{xunicode}
\usepackage{xltxtra}

\usepackage{unicode-math}

\setmathfont{Asana Math} % base font
% \setmathfont{EB Garamond} % To use in the future?

\setmathfont[range=\mathup/{num,latin,Latin,greek,Greek},
             Ligatures=TeX, Numbers={Lining},
             SizeFeatures={
               {Size=-8, OpticalSize=8},
               {Size= 8-, OpticalSize=12}}]{EB Garamond}

\setmathfont[range=\mathit/{num,latin,Latin,greek,Greek},
             Ligatures=TeX, Numbers={Lining},
             SizeFeatures={
               {Size=-8, OpticalSize=8},
               {Size= 8-, OpticalSize=12}}]{EBGaramond12-Italic}

\setmathfont[range=\mathbfup/{num,latin,Latin,greek,Greek},
             Ligatures=TeX, Numbers={Lining}, FakeBold=1.5,
             SizeFeatures={
               {Size=-8, OpticalSize=8},
               {Size= 8-, OpticalSize=12}}]{EB Garamond}

\setmainfont[Mapping=tex-text,Numbers={OldStyle}]{EB Garamond}

\setsansfont[Numbers={OldStyle,Proportional},Scale=MatchLowercase]{Linux Biolinum O}
\setmonofont[Scale=MatchLowercase]{DejaVu Sans Mono}

\newcommand{\texto}{Este é um teste. 123456789. Aa Ee Ii Oo Uu ãà çÇ üÜ
  äÄ ëË õÕ © ðРߧ Åå ® £€ ¼½¾}

\usepackage{polyglossia}
%\setdefaultlanguage{english}
\setdefaultlanguage{brazil}


\title{Multilingual Math Test}
\author{Rogério Theodoro de Brito}


\begin{document}
\maketitle
\section{Teste}

Um teste. Este é um teste. Paul Erdös. Paul Erd\H{o}s. Otakar Borůvka. 1234567890

\begin{itemize}
  \item \emph{\texto}
  \item \textbf{\texto}
  \item \textsl{\texto}
  \item \texttt{\texto}
  \item \texttt{\emph{\texto}}
  \item \textsc{\texto}
\end{itemize}

\section{Projective Planes}

\emph{Projective planes} are a special case of block designs, where we
have $v > 0$ points and, as they are symmetric designs, $ b = v$ (which
is the limit case of Fisher's inequality), from the first basic equation
we get
\[
k = r,
\]
and since $\lambda = 1$ by definition, the second equation gives us
\[
v-1 = k(k-1).
\]

Now, given an integer $ n \geq 1$, called the \emph{order of the
  projective plane}, we can put $k = n + 1$ and, from the displayed
equation above, we have $v = (n+1)n + 1 = n^2 + n + 1$ points in a
projective plane of order $n$.

Since a projective plane is symmetric, we have that $b = v$, which means
that $b = n^2 + n + 1$ also. The number $b$ is usually called the number
of \emph{lines} of the projective plane.

This means, as a corollary, that in a projective plane, the number of
lines and the number of points are always the same. For a projective
plane, $k$ is the number of lines and it is equal to $n + 1$, where $n$
is the order of the plane. Similarly, $r = n + 1$ is the number of lines
to which the a given point is incident.

For $n = 2$ we get a projective plane of order $2$, also called the
\emph{Fano plane}, with $v = 4 + 2 + 1 = 7$ points and $7$ lines. In the
Fano plane, each line has $n + 1 = 3$ points and each point belongs to
$n + 1 = 3$ lines.

\section{Extensão de Corpos}

Seja $a$ um número complexo que satisfaça $x^3 - x + 1 = 0$. Seja $b =
a^2$.  Então nós temos que $a^3 + 1 = a$, isto é, $a^3 = a - 1$.  Como
$a^3 + 1 = a$ nós temos que
\begin{align*}
  (a^2)^3 - 2(a^2)^2 + a^2 - 1 &= a^6 -2a^4 + a^2 -1\\
  &= (a^3)^2 - 2a a^3 + a^2 - 1\\
  &= (a-1)^2 - 2a(a-1) + a^2 -1\\
  &= a^2 -2a + 1 - 2a^2 + 2a + a^2 -1\\
  &= 0.
\end{align*}
Portanto $b = a^2$ é raiz de
\begin{equation}
  \label{eq:b-req}
  x^2 - 2x + x -1.
\end{equation}
Como $b = a^2$, então $b$ está em $\mathbb Q(a)$. Logo, $\mathbb Q(b)
\subseteq \mathbb Q(a)$. Além disso, $\mathbb Q \subseteq \mathbb
Q(b)$. Agora $\mathbb Q = \mathbb Q(b)$? Não, pois \eqref{eq:b-req} não
possui raiz racional. Logo $\mathbb Q \subsetneq \mathbb Q(b) \subseteq
\mathbb Q(a)$. Como $\mathbb Q(a)$ tem grau $3$ sobre $\mathbb Q$, então
pelo teorema de Dedekind de graus (multiplicativos) de corpos, o grau
$[\mathbb Q:\mathbb Q(b)][\mathbb Q:\mathbb Q(a)] = 3$.  Como $\mathbb Q \ne
\mathbb Q(b)$, nós temos que $[\mathbb Q:\mathbb Q(b)] = 3$, de onde segue
que $\mathbb Q(b) = \mathbb Q(a)$.

Alternativamente, observe-se que $a^2 - a^4 = b - b^2$, pela definição
de $b$.  Como $a^3 = a - 1$, nós temos que $a^4 = a a^3 = a(a-1) = a^2
-a$.  Logo, $a^4 = a^2 - a$, isto é, $a^2 - a^4 = a$, de onde segue que
$a = b - b^2 \in \mathbb Q(b)$ e, portanto, $\mathbb Q(a) \subseteq
\mathbb Q(b)$, ou seja, $\mathbb Q(a) = \mathbb Q(b)$.
\end{document}
